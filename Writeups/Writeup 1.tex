\documentclass{article}

\usepackage{natbib}
\usepackage[sc]{mathpazo}
\usepackage[T1]{fontenc}
\usepackage{amsmath}
\usepackage{amsfonts}
\usepackage{amssymb}
\usepackage{graphicx}
\usepackage[onehalfspacing]{setspace}
\usepackage{color}
\usepackage[margin=.75in, tmargin=0.71in, bmargin=0.71in]{geometry}
\usepackage{url}

\usepackage{appendix}
\usepackage{hyperref}
\usepackage{xcolor}
\usepackage{todonotes}
\usepackage{booktabs}
\usepackage{lscape}
\usepackage{caption}%
\usepackage{bbm}
\usepackage{comment}

\usepackage{longtable}

\usepackage{subcaption}

\usepackage{bookmark}

\usepackage{babel}
\usepackage[autostyle, english = american]{csquotes}
\MakeOuterQuote{"}

\title{Textual Analysis and Financial Statements}
\author{Isaac Liu}

\setlength{\parindent}{0pt}
\setlength{\parskip}{0.5em}

\hypersetup{
    colorlinks=true,
    linkcolor=black,
    filecolor=black,      
    urlcolor=blue,
    citecolor=black
}

% stattotex commands
\newcommand{\avgCompanyMentions}{97.78}
\renewcommand{\avgCompanyMentions}{98.66}
\renewcommand{\avgCompanyMentions}{98.66}
\renewcommand{\avgCompanyMentions}{98.66}
\renewcommand{\avgCompanyMentions}{98.63}
\renewcommand{\avgCompanyMentions}{98.63}

\newcommand{\numQuarters}{7333}
\newcommand{\numCompanies}{536}

\newcommand{\avgCallLength}{8,776.18}


\begin{document}

	\maketitle

    \section*{Introduction}

    high-level subject area info

    \citep{das_credit_2023}

    problem statement and question

    Company ratings and creditworthiness are important information for investors - not just institutional investors and financially sophisticated bondholders, but also stockholders, who may be wiped out completely in the event of bankruptcy.

    Are ratings based on hard numbers, or do company outlooks and sentiment also matter? Are they predictable?

    note credit rating data access is limited and our model can be used to interpolate

    In this project, we seek to explore whether incorporating the text of earnings calls improves predictions of corporate credit ratings.

    high level data description

    Though much literature has focused on financial statements and reports and credit ratings (CITATIONS HERE), our paper takes a relatively underexplored approach, instead incorporating earnings call transcripts. We believe calls offer a richer picture of a firm's financial prospects because they include two-way conversation between company management and financial analysts in form of a Q and A section. This section incorporates the broader beliefs and concerns of the financial community into our predictions. Prior work has recognized the 'wisdom of the crowd' in making predictions, and our approach fully accounts for it, going beyond the proclamations of a company's management.

    Calls also require less parsing - no looking for specific section of financial statement reports.

    roadmap
    we then

    \section*{Data and Exploratory Data Analysis}

    We combine a wide variety of data sources to support our predictions of credit ratings - combining rating data with company earnings calls, financial statement variables, and industry sector. In our final dataset, each observation represents a fixed quarter date (1/1, 4/1, 7/1, 10/1) for a company, with the company's most recent credit rating, earnings call and associated financial statement variables, and sector attached.

    Our scope of interest is publicly traded companies from 2010-2016 (a limitation due to the availability of credit rating data). The data is temporally unbalanced, with many companies entering the dataset in later years after they first receive an observable credit rating (Figure \ref{fig:obs-by-quarter-year}).

    \begin{figure}[h!]
		\centering
        \caption{Observations by Quarter and Year}
        \includegraphics[width=0.5\linewidth,keepaspectratio=true]{../Output/All Data EDA/Tabular EDA/all_data_fixed_quarter_dates_obs_by_year_quarter_no_title.png}
        \label{fig:obs-by-quarter-year}
	\end{figure}

    In all, we have \numQuarters \space quarters for \numCompanies \space unique companies.

    \subsection*{Credit Ratings}

    We make use of long-term credit rating issuances from S and P Rating Services, provided from a combination of two credit rating datasets downloaded in CSV and Excel format from Kaggle \citep{gewerc_corporate_2020,makwana_corporate_2022}. Each issuance be a change in rating (upgrade, downgrade) or reaffirmation - they occur at ad-hoc intervals. We reshape these rating issuances to a dataset of ratings for each company on each fixed quarter date by creating a rating end date variable that is the date of the next issuance, and joining a list of the fixed quarter dates on the condition that the fixed quarter date is between the issuance date and the end date.

    Figure \ref{fig:credit-ratings} shows the distribution of rating grades used in our final dataset. Finer grades (+, -) are sometimes assigned by agencies, but these grades were removed for this project. Ratings of BBB and above are considered investment grade - these bonds carry empirical one-year default rates of ~0 to 1\%. Ratings below that are classified as junk, with default rates from 1 to 30, 40, or even 50\% for some years \citep{s_and_p_global_ratings_s_2024}. Most company-quarters have ratings around the BBB threshold, with very few cases on the extreme ends of the spectrum.

    \begin{figure}[h!]
		\centering
        \caption{Credit Ratings}
        \includegraphics[width=0.5\linewidth,keepaspectratio=true]{../Output/All Data EDA/Tabular EDA/Distribution of Rating Issuances_no_title.png}
        \label{fig:credit-ratings}
	\end{figure}

    \subsection*{Earnings Calls}

    Our earnings call data comes from the Financial Modelling Prep API \citep{financial_modelling_prep}, a trusted source widely used in industry. We remove all calls that happened more than 250 days prior and after the year and quarter they are supposed to discuss the results from. Including both prepared remarks and analyst Q and A sessions, the overall average call length in our final data stands at \avgCallLength \space words.

    \begin{figure}[h!]
		\centering
        \caption{Number of Words in Earnings Calls}
        \includegraphics[width=0.5\linewidth,keepaspectratio=true]{../Output/All Data EDA/NLP EDA/all_data_num_words_distribution_no_title.png}
	\end{figure}

    \subsection*{Financial Statements}

    Our financial statement variables are also retrieved using the Financial Modelling Prep API. We make use of items from company balance sheets, cash flow statements, and income statements, as well as company market capitalization. We include 124 variables in total, such as revenue, total liabilities, net income, EBITDA. % for a list, see appendix
    
    Limit to items reported in USD
    
    Winsorizing: Check for items mis-multiplied by 1,000 in parsing - if last digits are “000.00” and item is above or below 2.5\% and 97.5\% quantile, divide by 1,000

    Tests to ensure the value in income statement and balance sheet are consistent with each other.

    Construct Altman Z-score

    \begin{figure}[h!]
		\centering
        \caption{Altman Z-Score}
        \includegraphics[width=0.5\linewidth,keepaspectratio=true]{../Output/All Data EDA/Tabular EDA/altman_z_score_all_data_no_title.png}
	\end{figure}    

    \subsection*{Sector}

    GCIS developed by S and P

    Obtained from Kaggle with supplementary manual lookup

    CSV and Excel format

    \begin{figure}[h!]
		\centering
        \caption{Firms by Sector}
        \includegraphics[width=0.5\linewidth,keepaspectratio=true]{../Output/All Data EDA/Tabular EDA/all_data_fixed_quarter_dates_firms_by_sector_no_title.png}
	\end{figure}

    sectoral imbalance    

    \subsection*{Quality Control}

    quality control
    code review of all data cleaning code
    numerous investigations

    date gaps investigations

    company dropout and the company that leaves

    \section*{NLP Features}

    Average call length

    \begin{figure}[h!]
		\centering
        \caption{Average Call Length by Credit Rating}
        \includegraphics[width=0.5\linewidth,keepaspectratio=true]{../Output/All Data EDA/NLP EDA/all_data_call_length_by_credit_rating_no_title.png}
	\end{figure}

    outliers and errors

    correlations and patterns

    identification of good machine learning methods

    \section*{Modelling}

    Our overall model architecture is of the form

    \begin{equation*}
        \text{Predicted Credit Rating} = f(\text{Financial Statement Variables}, \text{Sector}, \text{NLP Features})
    \end{equation*}

    functions began with logistic regression

    XXX logistic regression predictors

    multinomial, balanced class weights, l1 penalty

    table of predictions

    fitting and output

    assumptions
    
    interpretation

    \section*{Next Steps}

    Ensembling and Auto-ML

    more classifiers
    
    first steps using AutoML

    a good starting point for diving deep on more algorithms

    algorithms and accuracy from them

    outputted feature importance

    Graph Neural Network incorporating the relationships between companies, trained end-to-end with both tabular financial data and NLP features
    
    % REDO for companies that aren't the company giving the call...
    %On average, each earnings call has \avgCompanyMentions company mentions (though we have not yet distinguished between mentions of the company giving the call and other companies).
    % \begin{figure}[h!]
	% 	\centering
    %     \caption{Company Mentions}
    %     \includegraphics[width=0.5\linewidth,keepaspectratio=true]{../Output/All Data EDA/NLP EDA - NER on Company Names/Company Mentions Distribution No Title.png}
	% \end{figure}

    Fine tune the pre-trained LLMs for NLP feature construction
    
    \clearpage
    \newpage

    \bibliographystyle{aea}
    \bibliography{Stat-222-Capstone}

    % \clearpage
    % \newpage

    % \appendix

    % \section*{Appendix}

\end{document}
